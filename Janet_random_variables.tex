\section{Binomial and Hypergeometric}\smallskip \hrule height 2pt \smallskip
Note: this is for objects that are indistinguishable.  You can distinguish a red fish from a blue fish but you can't tell whether a red fish's name is "Mary" or "Gary". 
\hfill \\

\subsection{Binomial distribution/"probability"}
	\begin{description}
		
		\item[Binomial Distribution] ("Probability" in 10/12 lecture). A series of \textbf{independent trials}, each resulting in one of \textbf{two} possible outcomes, "success", or "failure".  So yes replacement if it keeps trials independent, and can't have three possible outcomes. 
			\[ P(k \mbox \ successes) = {{n}\choose{k}}p^k(1-p)^k \mbox{, for \ } k = 0, 1, \dots, n \]
		Note this is the number of ways of getting $k$ indistinguishable objects in n tries.  Then you multiply by the probability, $p^k$, of getting that number $k$, and the probability $(1-p)^k$ of getting the rest.  $P(\mbox{2 heads out of three for a coin that gives heads 3/4 of the time}) = (3)*(0.75^2)*(0.25^1)$ \hfill \\
		\hfill \\
	\end{description}

\subsection{Hypergeometric Distribution}
	\begin{description}
		\item[Hypergeometric Distribution]  For \textbf{no replacement}.  Draw $n$ chips from an urn that has $r$ red chips, $w$ white chips, and $r + w = N$.   $n$ is the total number of chips you will draw; $k/n$ will be the fraction you draw that are red. Draw chips without replacement.  Note k must be $<$ r or the problem doesn't make sense and something like ${{r} \choose {k}} = {{1} \choose {5}} = ?!?!$ can happen.  
			\[ P(\mbox{k red chips chosen}) = \frac{ {{r} \choose {k}} {{w} \choose {n-k}} }{{{N}\choose{k}}} \] % \frac{{{r}\choose{k}}{w \chose {n-k}}}{N \chose k} \]
			\[ \frac{\mbox{(ways to draw \ k red from r)}*\mbox{(ways to draw n-k non-red from r)}}{\mbox{total number of combinations of chips you can draw}} \] \hfill \\
		Note: this is the same as	{\tiny pg 111}
		\[ = \frac{ {{n} \choose {k}} ({{_r}P_{k}}) ({{_w}P_{n-k}}) }{{{_N}P_n}} \] 
			\hfill \\
		Note that if you sample a very small fraction of the fish pond you will recover the binomial distribution (because replacement vs. no replacement isn't so important). \hfill \\
		\hfill \\
		Say you have $1, 2, 3, \dots, t$ types of objects with numbers $n_1, n_2, \dots, n_t$ in an urn.
		If you want to chose $k_1$ objects of  type 1, $k_2$ objects of  type 2, $\dots$ $k_t$ objects of  type $t$.\
		Then the number of objects is $N = n_1 + n_2 + \dots + n_t$, the number you are selecting is $n = k_1+ k_2 + \dots + k_t$, and  the formula is   {\tiny q 3.2.35, pg 118}
		\[  \frac{ {{n_1}\choose{k_1}} {{n_2}\choose{k_2}} \dots {{n_t}\choose{k_t}} }{ {{N}\choose{n}} }  \]
		This is different than the formula I derived.  I was multiplying the probabilities separately, and updating N each time.  Is this an overestimation? 		
	\end{description}
	
\section{Random Variables} 

\begin{center}
\begin{tabular}{ c c c c c }
 \textbf{R.Var.} & \textbf{pdf} & \textbf{cdf} &   \\ 
 	discrete &   pmf (m $=$ mass) & cdf \\
	continuous & pdf (d $=$ density) & cdf
 \end{tabular}
\end{center}
\hfill \\
A probability mass function (pmf) is a function that gives the probability that a discrete random variable is exactly equal to some value.
A probability mass function differs from a probability density function (pdf) in that the latter is associated with continuous rather than discrete random variables; the values of the latter are not probabilities as such: a pdf must be integrated over an interval to yield a probability.
%\begin{center}
%\begin{tabular}{ c c c c c }
 %\textbf{R.Var.} & \textbf{density} & \textbf{cumulative} &   \\ 
 %	discrete & pmf  or pdf & cdf \\
%	continuous & pdf & cdf
 %\end{tabular}
%\end{center}
%\hfill \\


\subsection{Discrete probability function}
Suppose S is a finite or countably infinite sample space.  Let $p$ be a real-valued function defined for each element of $S$ such that: {\tiny pg 119} \hfill \\
(a)  \[ 0 < p(s) \mbox{ for each \ } s \in S  \] 
(b)  \[  \sum\limits_{s \in S} p(s) = 1 \]
Then $p$ is said to be a \textit{discrete probability function} \hfill \\
\hfill \\
Once $p(s)$ is defined for all $s$, then you can say the probability of any event $A$ is the sum of the probabilities of the outcomes comprising $A$: 
\[  P(A) = \sum\limits_{s \in S} p(s) \]
This function satisfies the probability axioms of Section 2.3.   \hfill \\

Note: you can still have an infinite number of outcomes in the sample space, as long as the probability of all outcomes sums to one.  E.g. probability of getting heads on an odd numbered coin toss has an infinite number of events but if you sum the two sums (got it on odd, got it on even), you get a sum of 1.  {\tiny pg 120}

\subsection{Discrete random variable}
3 Lecture Axioms: {\tiny 10/14}:   $\Omega \rightarrow {\cal X} \subseteq $ R   \hfill \\  
(The sample space is ${\cal X}$, which is real.)  Pg 119 has a similar set of facts, that are presented a little differently. 
 \hfill \\  
	\[  P(X=x) \geq 0 \mbox{, \ \  }  P(X=x) > 0  \mbox{ if } x \in {\cal X}  \]
%(book version (pg 119)): $ 0 \leq p(s)$ for each $s \in S$. ($S$ is a finite or countably infinite sample space. P is a real -valued function defined over eac element of S) 
	\[  P(\Omega) = 1 \mbox{, so \ }  \sum\limits_{x \in \cal X} P(x= {\cal X}) = 1\]
$E_1$, $E_2$, $\dots$, $E_k$ are disjoint (non-overlapping). 
	\[ P(E_1 \cup E_2 \cup \dots \cup E_k) =  \sum\limits_{i=1} P(E_i) \mbox{;} P(x \in B) = \sum\limits_{{\cal X} \in B} P(x = {\cal X}) \]
We have done this.  E.g. P(X is 2, 3, or 4) $= P(X=2)$ + $P(X=3)$ + $P(X=4)$
	%\[  E_1, E_2, \dots, E_k \mbox{ are disjoint (non-ovaerlapping).} P(E_1 \cup E_2 \cup \dots \cup E_k) =  \sum\limits_{i=1} P(E_i) \]
	%\[  P(\Omega) = 1, so  \sum\limits_{i=1}^k P(E_i)  \mbox{ if } x \in {\cal X} \]
	
     \hfill \\  
  \hfill \\  
A function whose domain is a sample space $S$ and whose values form a finite or countably infinite set of real numbers is called a \textbf{discrete random variable}.  We denote random variables by uppercase letters, often X or Y.  {\tiny page 123}  \hfill \\
Example: sum of the values of two die faces.  E.g. value of die 1 is $X_1$, value of die 2 is $X_2$, value of sum is $X = X_1 + X_2$    \hfill \\  
  \hfill \\  
  
\textbf{The Probability Density Function}.  Associated with every discrete random variable X is a \textit{probability density function} (or \textit{pdf}), denoted $p_x(k)$ where 
	\[  p_X(k) = P({s \in S \mid X(s) = k})  \]
That's often written as $ p_X(k) = P(X = k) $.  \hfill \\  
Note: the binomial distribution is such an example:   (So is hypergeometric.) 
	\[  p_x(k) = P(k \mbox \ successes) = {{n}\choose{k}}p^k(1-p)^k \mbox{, for \ } k = 0, 1, \dots, n \]

\subsection{Cumulative Distribution Function} \textbf{(Discrete)}
You might want $P(x \leq X \leq t) = P(X \leq t) - P(X \leq s - 1)$  {\tiny pg 127}  \hfill \\  
\textbf{Cumulative distribution function}:  Let $X$ be a discrete random variable.  For any real number $t$, the probability that $X$ takes on a value $\leq t$ is the \textit{cumulative distribution} (cdf) of $X$ [written $F_X(t)$].
	\[ F_X(t) = P({s \ in S \mid X(s) \leq t}) \]
or more simply 
	\[  F_X(t) = P(X \leq t)  \]
then {\tiny 10/19 lecture}
	\[  P(a \leq X \leq b) = F_X(b) - F_X(a)  \]
	
\subsection{Continuous random variables}
A probability function $P$ on a set of real numbers $S$ is called \textbf{continuous} if there exists a function $f(t)$ such that for any closed interval $[a, b] \subset S$, $P([a, b]) = \int_{a}^{b} f(t)dt$. \hfill \\  
Kind of obvious, but all values of the function must be more than zero.  \hfill \\  

\subsection{Continuous probability density functions}
Density functions are \textbf{not a probabiliy}.  {\tiny (11/4 lecture)}  \hfill \\ 
\hfill \\  

{\tiny (pg 135)} Let $Y$ be a function from a sample space $S$ to the real-numbers (takes values of real numbers; put in something and you get out a real number).  The function $Y$ is called a \textit{continuous random variable} if there exists a function $f_Y(y)$ such that for any real numbers $a$ and $b$ with $a < b$:
	\[  P(a \leq Y \leq b) = \int_{a}^{b} f_Y(y)dy \]
The function $f_Y(y)$ is the \textbf{\textit{probability density function (pdf)}} for $Y$.  Think of "density" as corresponding to how much height there is over the x-axis when you plot $f_Y(y)$ versus $y$.  \hfill \\

\hfill \\ 
\textbf{The value of $f$ can be greater than 1}.  {\tiny See 10/28 lecture.}   If $f_X(x) = c$ for $ 0 \leq x \leq 1/2$ then the area has to $= 1$ so $c = 2$.  So $p_X(x) = 2$. 

\subsection{Continuous cumulative distribution functions}
As in the discrete case, the \textit{cumulative distribution function (cdf)} is defined by $F_Y(y) = P(Y \leq y)$: {\tiny (pg 136)}
	\[  F_Y(y) = P(Y \leq y) = \int_{- \inf}^{y} f_Y(t)dt \]
Also written as  {\tiny (Definition 3.4.3: pg 137)}
	\[  F_Y(y) = \int_{- \inf}^{y} f_Y(r)dr = P({s \in S \mid Y(s) \leq y}) = P(Y < y) \]
The cdf in this case is an integral of $f_Y(y)$.  Note that now we have $f_Y(t)$ or $f_Y(r)$, not $f_Y(y)$.  We are still integrating over the x-axis ($y$) but we can't have $y$ in $dy$ and in the integration limits.   \hfill \\  
Also note the derivative of the cdf is the pdf:
	\[  \frac{d}{dy}F_Y(y) = f_Y(y) \]
 \hfill \\  
  \hfill \\  
\subsection{Indepenence}	
If X, Y are independent, $P_{X,Y}(x,y) = P_X(x) \cdot P_Y(y)$ for all $x$, $y$.
Note there are cases when $E(XY) = E(X)E(Y)$ but this does \textbf{not} imply independence. {\tiny (See Lec 11/4)}
 \hfill \\  	
 \hfill \\  
 
\subsection{Expected Values}
Let $X$ be a discrete random variable with probability function $p_X(k)$.  The \textit{expected value} of $X$ is denoted $E(X)$ (or sometimes $\mu$ or $\mu x$ and is given by: {\tiny pg 140}   \hfill \\  
\textbf{Discrete:}
	\[  E(X) = \mu = \mu x = \sum\limits_{\mbox{all } k} k*p_X(k)  \]
\textbf{Random:}	  \hfill \\  
	\[  E(Y) = \mu = \mu y = \int_{- \inf}^{\inf} y*f(y)dy  \]
\textbf{binomial:} {\tiny pg 141, ~10/16 TA lecture} 	  \hfill \\
	\[ E(X) = np =  \sum\limits_{k=0}^{n} k \cdot p_X(k) = \sum\limits_{k=0}^{n}k \cdot {{n}\choose{k}}p^k(1-p)^k \]  
\textbf{hypergeometric}: for selecting $n$ balls from $r$ red balls and $w$ white balls{\tiny pg 143, ~10/16 TA lecture}  \hfill \\  
	\[ E(X) = \frac{rn}{r+w} \]
turns out the same if you substitute the proportion of red balls:
	\[ E(X) = n \cdot p \mbox{ for } p= \frac{r}{r+w} \]
\hfill \\  	

\subsection{The median}
If $X$ is a discrete random variable, the median, $m$, is that point for which $P(X < m) = P(X>m)$.  In the event that $P(X \leq m) = 0.5$ and $P(X \geq m')$, the median is defined to be the arithmetic average, $(m + m')/2$.  \hfill \\
If $Y$ is a continuous random variable, its median is the solution to the integral equation $\int_{- \inf}^{m} f(y)dy = 0.5$  \hfill \\
If a random variable's pdf is symmetric, both $\mu$ and $m$ will be equal.  

\subsection{Variance}
{\tiny (pg 157; Sect3.6)}  
The variance of a random variable is the expected value of its squared deviations from $\mu$.  
Visually, it is the moment of inertia. \hfill \\  
Always positive. {\tiny -E.T. MT2 review} \hfill \\  
\hfill \\  
Discrete: $Var(X) = \sigma ^2 = E[(X-\mu)^2] = \sum \limits_{\mbox{all } k} (k - \mu)^2 \cdot f_X(k)$  \hfill \\  
\hfill \\  
Continuous: if Y is continuous with pdf $f_Y(y)$:   \hfill \\
Var($Y$) = $\sigma ^2 = E[(Y - \mu)^2] = \int_{-\infty}^{\infty}(y- \mu)^2 \cdot f_Y(y) dy$   \hfill \\  
\hfill \\  
Discrete or continuous: $Var(W) = E(W^2) - (E(W))^2 = E(W^2) - \mu ^2$\hfill \\  
\hfill \\  

Variances always add (independence not required).  And, 
$Var(aX + bY) = a^2\cdot Var(X) + b^2\cdot Var(Y) $ {\tiny (page 189)} \hfill \\  
\hfill \\  

\textbf{Standard Deviation}: $\sigma$.  Square root of variance. \hfill \\  

\subsection{Misc.}
$F_X(x) = P(X \leq x)$  \hfill \\   
$p_X(x) = P(X=x) = P(X \leq x) - P(X < x)$ (discrete)  {\tiny (lec 10/28)}  \hfill \\  
 \hfill \\  
 PDFs can have values greater than 1. 

\section{Multiple random variables} 
\textbf{Double sigma notation}: If you have two $\sum$s, you have a 2D grid of values.  (\href{<https://www.youtube.com/watch?v=BK7OEt9AHIw>}{simple video}).  If the two dimensions start and stop at numbers (or maybe $\infty$) then you have a value in every position of the grid.  There can be cases, however, where the values of the two variables are interdependent (\href{<http://math.stackexchange.com/questions/490723/double-summation-switch>}{example}). 

Example with coefficients that depend on both indices (hasn't come up yet): 
Consider real numbers $a_{ij}$, where $i$ ranges from 1 to 3, and $j$, from 1 to 2. We thus have the following equality:
$$\sum_{i=1}^3 \sum_{j=1}^2 a_{ij}=\sum_{i=1}^3(a_{i1}+a_{i2})=(a_{11}+a_{12})+(a_{21}+a_{22})+(a_{31}+a_{32}).$$


\textbf{Vocab}:  \hfill \\
\textbf{joint density} - pdf for a joint distribution. \hfill \\
\textbf{univariate distribution} - pdf of distribution with exactly 1 random variable \hfill \\
\textbf{marginal distribution} - \hfill \\
\hfill \\
If you have a joint distribution with 2 random variables, the two marginal distributions are univariate distributions \hfill \\
\hfill \\

See Elizabeth's single-page PDF for lots of formulas. 
Key ones:  {\tiny 10/28 lecture}

\textbf{One Dimension:}  $ P(a \leq b) = P(X \leq b) - P(X \leq a) = F_X(b) - F_X(a)$   \hfill \\
\textbf{Two Dimensions:} $ P(a_1 < x \leq a_2 , b_1 < y \leq b_2) = F_{X,Y}(a_2, b_2) - F_{X,Y}(a_1, b_2) - F_{X,Y}(a_2, b_1) + F_{X,Y}(a_1, b_1) $

\textbf{Marginal CDFs:} If you have $F_{X,Y}(a,b)$, \hfill \\ 
$F_X(a) = P(X \leq a, y < \infty)$ = $F_{X,Y}(a, \infty)$   \hfill \\ 

See Elizabeth's PDFs for a lot more formulae. 

%\begin{center}
%\begin{tabular}{ c c c }
% \textbf{R.Var.} & \textbf{1 variable/dimension} & \textbf{2 variables/dimensions}  \\ 
 %joint CDF  & $  \begin{tabular}{@{}c@{}} $ P(a \leq b) = P(X \leq b) - P(X \leq a) $  \\  $= F_X(b) - F_X(a) $ \end{tabular} $ &  
%  			\begin{tabular}{@{}c@{}} $ F_{X,Y}(a , b) = P(X \leq a \cap y \leq b) $  \\  $= F_X(b) - F_X(a) $ \end{tabular}  \\  
%\end{tabular}
%\end{center}
%\hfill \\


\subsection{Expected values for functions of random variables}
\textbf{NOTE: Expectations by default don't have X, Y in them.}  

{\tiny (pg 150; Lecture 10/14)}  Suppose $X$ is a discrete random variable with pdf $p_X(k)$.  Let $g(X)$ be a function of $X$.  Then the expected value of the random variable $g(X)$ is given by: \hfill \\
	\[ E[g(X)] = \sum\limits_{\mbox{all k}} g(k) \cdot p_X(k) \]
Provided that $\sum\limits_{\mbox{all k}} | g(k)| \cdot p_X(k) < \inf$
\hfill \\
\hfill \\
If $Y$ is a discrete random variable with pdf $f_Y(y)$, and if $g(Y)$ is a continuous function, then the expected value of the random variable $g(Y)$ is: \hfill \\
	\[ E[g(Y)] = \int\limits_{- \inf}^{\inf} g(y) \cdot f_Y(y)dy \]
Provided that $E[g(Y)] = \int\limits_{\inf}^{\inf} | g(y)| \cdot f_Y(y)dy < \inf$  \hfill \\
\hfill \\

These two definitions highlight the difference in nomenclature.  See page 127 for definition of discrete cdf.  See page 135 for definition of continuous cdf 
\begin{center}
\begin{tabular}{ c c c c c }
 \textbf{R.Var.} & \textbf{x-axis} & \textbf{pdf notation} & \textbf{cdf notation}   \\ 
 $\mbox{\tiny discrete  } X$     & $k$ & $p_X(k)  \mbox{ \tiny or }  P(X=k) $ & $F_X(k) \mbox{ \tiny or } P(X \leq k)  $  \\  
$\mbox{\tiny continuous  } Y$ & $y$ & $f_Y(y)$                                          & $F_Y(y) \mbox{ \tiny or } P(Y \leq y)  $ 
\end{tabular}
\end{center}
\hfill \\


 \begin{center}
\begin{tabular}{ c c c c c }
 \textbf{R.Var.} & \textbf{pdf} & \textbf{cdf}  \\ 
 {discrete  } $X$    & \begin{tabular}{@{}c@{}} $ p_X(x) = P(X=x)$   \\  $= P(X \leq x) - P(X <x)$ \end{tabular}         & $F_X(x)  =  \sum\limits_{z \leq x}p_X(z)$  \\  
 {continuous  } $Y$ & $p_X(x) = 0$ {\tiny (at one point)}  & $F_Y(y) = \int_{-\infty}^{y}f_Y(y)$ \\  
\end{tabular}
\end{center}
\hfill \\
 
\textbf{NOTE:}  Need to have intuitive understnading that $P$s are probabilities and $F$s are cdfs.  Can't just assume capitol letters are cdfs.
%When we start talking about transforms of random variables, you can get $P$ that isn't CDF.  
E.g. if $Y = X^3$ then the $P$s in $P(Y = y) = P(X^3 = y) = P(X = y^{1/3}) $ are probability, not cdf.  Note that there is no subscript in this case.  
%But that it looks rather like $P(X=k)$ in table above.   % $P(X) \mbox{ \tiny or } P(X = y)$

\hfill \\
Say your random variable is the number on a die face.  Say your function of that random variable is the square of that number.  If you want the expectation of the square, you can sum the products of the squares with the probabilities of the random variable values.  \hfill \\
\hfill \\
\textbf{Theorem} {\tiny (10/14 Lecture \& similar on pg 150)}:  \hfill \\
let $Y = g(X)$.  Since:
	\begin{align*}
		\mathbb{E} (Y) &=  \sum\limits_{y} yP(Y=y)   \\
			& = \sum\limits_{y} y \sum\limits_{x:g(x)=y} P(X = {\cal X})  \\
			& \mbox{note: above is a sum over the $y$ values,} \\ 
			& \mbox{and a sum over the $x$s that can give those y values.} \\
			& \mbox{There can be multiple $x$s that give a particular $y$.} \\
			& = \sum\limits_{y} \sum\limits_{x:g(x)=y} g(x)P(X=x) \\
			& \mbox{Drop $y$ sum b/c now everything is in terms of $x$} \\
			& \mbox{You are summing over all the events in a set that }   \\
			& \mbox{are indexed by y.  But when you take the sum over y, }  \\
			& \mbox{you say let's sum over all the different outcomes. } \\
			& = \sum\limits_{x \in {\cal X}} g(x)P(X=x) = \sum\limits_{x \in {\cal X}} g(x)P({\cal X}) \\ 
			& \mbox{This is a sum over all the $x$s in the whole set of outcomes (${\cal X}$)} 
	\end{align*} 
Note that this works for $g(Y) = Y^2$.  $E(Y^2) =  \int_{-\inf}^{\inf} Y^2 \cdot f_Y(y)dy  $.  {\tiny 10/21 review lecture}  \hfill \\
Loop over all the values the random variable takes (all $x$s in ${\cal X}$), and sum (probability of that $x$)*(the function applied to that $x$).  \hfill \\
 \hfill \\
 
\textbf{Addition of functions of random variables} 
Adding sums of expectations always works; doesn't require independence. { \tiny (10/14 lecture)}
	\begin{align*}
		\mathbb{E} (g_1(x) + g_2(x)) &=   \sum\limits_{x \in {\cal X}} (g_1(x) + g_2(x))p(x) \\
			&= \sum\limits_{x \in {\cal X}} g_1(x)p(x) + \sum\limits_{x \in {\cal X}} g_2(x)p(x) \\
			&=\mathbb{E}  (g_1(x)) + \mathbb{E} (g_2(x)) 
	\end{align*} 
Simplest case: $E(X + Y) = E(X) + E(Y)$, valid even if X is not statistically independent of Y. \hfill \\
 \hfill \\
 
 
Example $\mathbb{E}  (a(x) + b)$:
	\begin{align*}
		\mathbb{E} (a(x) + b) &=  \mathbb{E}  (a(x)) + \mathbb{E} (b) \\
			& = a\mathbb{E} (x) + b
	\end{align*} 
Example $\mathbb{E}  (x(x-1))$:
	\begin{align*}
		\mathbb{E} (x(x-1)) &= \mathbb{E} (x^1 - x)) \\
			& =  \mathbb{E}  (x^2) - \mathbb{E} (x)    \\
			& \mbox{this might be easier to evaluate}
	\end{align*} 
Example $\mathbb{E}  (X^2 + Y^2)$:  {\tiny HW 5, Q 3.9.10}
	\begin{align*}
		\mathbb{E} (X^2 + Y^2) &= \mathbb{E}(X^2) + \mathbb{E}(Y^2) \\
	\end{align*} 
	
\section{Examples}
\textbf{Different ways shapes can come up:} \hfill \\
\underline{Probability of landing below some x value within a weird 2D shape.}  {\tiny 10/19 lecture}  \hfill \\
 If you have a diamond with points on the X and Y axes, but the top/bottom is cur off, what is probability of landing at some x?
 (What is $f_X(x)$?)  
 You can't just derive something because X is less probable as you go out to the points.  
 Use the classic trick: start with the CDF. v
 $F_X(x) = P(X \leq x) = 1 - F_X(x) = P(X \geq x) $
 \hfill \\
 \hfill \\
\underline{Probability of a subset of events.}  \hfill \\
* E.g. $Y > 2X$.  If you have a $f_{X,Y}(x,y)$ and you want to know what the probability is that $Y > 2X$ you can draw the shape (which should be smaller than the normal ranges of X and Y) and integrate both variables. \hfill \\
Integrate one using its two obvious limits and the other one using the relationship to the other variable.  {\tiny (E.g. Example 3.7.4 pg 165)}.  \hfill \\
Or if the distribution is uniform for the range of X, Y your $P(X,Y \mbox{ in some subset of X and Y values}) = $ fraction of the area.  \hfill \\
   \hfill \\

\underline{A shape imposing restrictions on allowable values of X or Y.}   (Indicator function multiplied on) \hfill \\
* E.g. $f_{X,Y}(x,y)=1/x, \mbox{ for } 0 \leq y \leq x \leq 1$ {\tiny (problem 3.7.20b)}.  If you want $f_X(x)$ or $f_Y(y)$ you have to use the $y <x$ relationship in the integral bounds.   Be careful with the integral bounds when getting cdf $F$s and expectations, too!  \hfill \\
* Or, problem 3.7.22 has $f_{X,Y}(X,Y) = 2e^{-x}e^{-y}$ but it is only valid for $y > x$ and 0 otherwise.  
Here the indicator function is ruining the independence.  I.e. the allowable shape is affecting the sample space.  
This will affect the CDF (e.g. integrate y from 0 to x), and the expectation.  It will also affect the marginal probabilities for either X or Y.   Use the $y < x$ bound (or $x > y$ bound) in each integration.  \hfill \\
\hfill \\ 

 \underline{A shape that represents a function of one or more random variables. }  \hfill \\
** E.g. area of triangle formed by (X,0), (0,Y), (0,0) for X, Y uniform on [0,1].
In this case the shape is only representing an aggregation of the variables.  It does *not* represent restricting values of either (e.g. $X < Y$).   
Then $W = (1/2)XY$.  X,Y uniform $ \rightarrow$ $E(XY/2) = (1/2)\cdot \mathbb{E}(X)\mathbb{E}(Y)$.  What is $f_{X,Y}(x,y)$ in this case?  Independence allows us to know $ f_{X,Y}(x,y)  \propto f_X(x)f_Y(y)$ since the limits don't affect each other (both valid over [0,1]).  \hfill \\
\hfill \\
     
 A shape that represents a function of one or more random variables.   \hfill \\
 \textbf{NO.  this isn't a shape.  It is a line.}
* $Y = X^2$ Here we could draw a line $Y = X^2$.  If we assume X is uniformly distributed, $f_Y(y) = c \cdot y^2$.   \hfill \\
** Now say X is not uniform.  Instead $f_X(x)=3x^2$ for $0 < x < 1$ then $F_Y(y) = P(Y \leq y) = P(X^2 \leq y) = P(X \leq y^{1/2}) = F_X(y^{1/2})$.  
We don't have $F_X(x)$ but can get it; it is $x^3$.  
Then $F_Y(y) = F_X(y^{1/2}) = (y^{1/2})^3 = y^{2/3}$ and $f_Y(y) = \frac{2}{3} y^{1/3}$  
We can get $\mathbb{E}(Y)$ two ways now.  \hfill \\
(Way 1:) This is always true: $E(Y)=E(X^2)$.  
We can use independence so $E(Y)=E(X^2)=(E(X))^2$.  
And $E(X)=\int_0^1 x \cdot f_X(x) = \int_0^1 x \cdot 3x^2 = 3/4 $.  Square that to get $9/16$.     \hfill \\
(Way 2:) Or, $E(g(x)) = \int_0^1 g(x) \cdot f_X(x)  = \int_0^1 x^2 \cdot 3x^2 dx$      \hfill \\
\hfill \\


-- ?? A shape restricting values of one or more variable(s).  Say    \hfill \\
	
\section{Joint Density}

$F_{X,Y}(a, b) = P(x \leq a \cap y \leq b) $.  \hfill \\
The $ \cap $ is often implicit: $P(x \leq a, y \leq b)$  {\tiny (10/28 lecture)}
 \hfill \\
 \hfill \\
 To get marginal: $F_X(a) = P(x \leq a) = P(X \leq a, y \leq \inf)$  {\tiny (10/28 lecture)}  \hfill \\
$ = F_{X,Y}(a, \inf)$   \hfill \\
  \hfill \\

If you are given $f_X(x)$ and $f_Y(y)$ you can't know the joint distribution *unless* you are told they are independent.  
 There are many univariate distributions that can lead to a particular joint distribution if the two variables are not independent.  {\tiny (11/3 TA lecture)}  \hfill \\
  \hfill \\


\textbf{Independence of random variables}.  {\tiny (page 175)} \hfill \\
Basic: Events $E_1$ and $E_2$ are independent iff $P(E_1 \cap E_2) = P(E_1) \cdot P(E_2)$  \hfill \\
 \hfill \\
For Random Variables: $E_1 \equiv \{ X \in A \}$, $E_2 \equiv \{ Y \in B \}$.  
Then if for \textit{all} subsets of $A$ and $B$, $P(X \in A, Y \in B) =P(X \in A) \cdot P(Y \in B) $ 
But if you had to check for every single A nd B it wouldn't be very useful.  
So, \textbf{Discrete Case}: {\tiny 10/30 lecture} \hfill \\  
\begin{align*}
	\mbox{Take } & A=\{X\}, B=\{Y\}  \\
	 \mbox{$X$, $Y$ independent: } & P(X=x, Y=y) = P(X=x) \cdot P(Y=y)   \\
	P(X \in A, Y \in B) &=  \sum\limits_{x \in {\cal A}} \sum\limits_{y \in {\cal B}} p_{X,Y}(x,y) \\
	&=  \sum\limits_{x \in {\cal A}} \sum\limits_{y \in {\cal B}} p_{X}(x)  \cdot p_{Y}(y) \\
	&= ( \sum\limits_{x \in {\cal A}} p_{X}(x)) ( \sum\limits_{y \in {\cal B}}   \cdot p_{X}(y) )\\
	&= P(X \in A) \cdot P(Y \in B)
\end{align*}
\hfill \\

\textbf{10/30 Independence facts} (10/30 lecture)  \hfill \\
Probabilities don't depend on each other: $P(X \leq x, Y \leq y) = P(X \leq x) \cdot P(Y \leq y)$ \hfill \\
CDFs don't depend on each other {\tiny (pg 174)}: $F(X \leq x, Y \leq y) = F(X \leq x) \cdot F(Y \leq y)$ \hfill \\
Also: $f_{X,Y}(x,y) \propto g_1(x)g_2(y)$ \textbf{BUT} limits that tie X and Y can make this break.  Recall the 10/30 example:  $0 < y < x < 1$ for $f_{X,Y}(x,y)=cxy$.  $cxy$ looks separable; be careful!

\hfill \\
\textbf{notation example:}
$f_{X_1, X_2 , \dots, X_n}(x_1, x_2 , \dots, x_n) = g_1(x_1) g_2(x_2) \dots g_n(x_n)$. \hfill \\
\hfill \\
Example: if $X_1$, $X_2$, and $X_3$ are independent random variables each with pdf $f_{X_i}(x_i) = 4x_i^3$. 
Then $f_{X_1, X_2, X_3}(x_1, x_2, x_3) = (4x_1^3)(4x_2^3)(4x_3^3) = 4^3 x_1^3 x_2^3 4x_3^3$ \hfill \\
\hfill \\

Example: {\tiny (10/30 lecture)} \hfill \\
$f_{X,Y}=c \cdot x \cdot y$ for $0 < x < 1$ and $0 < y < 1$.   
These are independent because you can pull the joint probability apart into single probabilities. \hfill \\
But if you have $f_{X,Y}=c \cdot x \cdot y$ for $0 < y < x  < 1$ they are not independent.  
You can tell right away because the bounds are dependent on each other.  You can also draw the $y=x$ line and shade the  $0 < y < x  < 1$ region to see.  \hfill \\
Erick: "Another way to say it is to remember that the function itself is c x y in some region and zero otherwise, and so the function itself is not a product."
\hfill \\
\hfill \\

\textbf{memoryless property}.  You are waiting for a phone call.  The probability of the phone ringing in the next minute is constant and doesn't depend on how long you have been waiting.  More formally: X = phone call time (?)   \hfill \\
$P(X \geq s + t \mid x \geq s) = P(x \geq t) $ \hfill \\
$ $  \hfill \\

\textbf{Combining random variables}
Example from 10/26 TA review lecture: 
Say you have 50 random variables that are uniform on $[0, 1]$.  $f_X1 = f_X2 = ... f_X50$ and independent.
The joint probability density function of these random variables is the product of the PDFs of the individual random variables.
The cdf is $F_X(x) = \{0 \mbox{ if } x < 0, x \mbox{ if } x \in [0, 1], 1 \mbox{ if } x > 1\}$.
Define $Y \equiv \mbox{max}(X_i)$.  Since they are independent, we know 
\begin{align*}
	P(Y \leq y) &= P(X_1 \leq y \mbox{ and } X_2 \leq y \mbox{ and }\dots \mbox{ and } X_{50} \leq y) \\
			& \mbox{\tiny independence allows multipication} \\
			&= P(X_1 \leq y) \cdot P(X_2 \leq y) \cdot \dots \cdot P(X_{50} \leq y) \\
			&= F_{X_1}(y) \dots F_{X_{50}}(y) =  y^{50} \\
		P(Y \leq y) &= y^{50} 
\end{align*}
{\tiny differentiate to convert from } $F_Y$ to $f_Y$:  $f_Y(y)=50y^{49}$  \hfill \\
\hfill \\

Example: Getting the expectation of one variable from a joint distribution {\tiny (10/30 lecture)}  \hfill \\
If you have $f_{X,Y}(x,y)$, and want $E(X)$, you can get $f_X(x) = \int\limits_{y} f_{X,Y}(x,y)dy$.   \hfill \\
Then You can get $E(X) = \int_{-\infty}^{\infty} x \cdot f_X(x)dx$.    \hfill \\
You can also get $F_{X,Y} = \int_{-\infty}^{x} \int_{-\infty}^{y} f_{X,Y}(u,v) du dv$ \hfill \\
Then you can get $F_X(x) = F_{X,Y}(x, y = \infty)$ \hfill \\

			
\section{Transforming/Combining Random Variables}

{\tiny (11/2 TA lecture)}  X and Y are independent: $f_X(x)= \frac{1}{2 \pi}exp(-x^2/2)$, $f_Y(y)= \frac{1}{2 \pi}exp(-y^2/2)$.   \hfill \\
Question: If $U = X^2 + Y^2$, what is $F_U(u)$?  \hfill \\
Answer: $f_{X,Y}(x,y)=f_X(x)f_Y(y) = \frac{1}{2 \pi} exp((-1/2)(x^2 + y^2))$ \hfill \\

\subsection{Location/Scale}
\textbf{$Y = X + b$:} 
\begin{align*}
	F_Y(y) &= P(Y \leq y) \\
			&= P(X \leq y-b)  \\
			&= F_X(y - b)  \\
			& \mbox{differentiate} \\
	      \mathbf{f_Y(y)} & \mathbf{= f_X(y-b)}
\end{align*}
?? Do you see why we are subtracting $c$, not adding it?  $f_{x+c}(t) = f_X(t-c) = P(X=t-c) = P(X+c=t)$  {\tiny (11/2 TA lecture)}

\subsection{Scaling Variables}
\textbf{$Y = aX $}: and $a > 0$. 
\begin{align*}
	F_Y(y) &= P(Y \leq y) \\
			&= P(aX \leq y)  \\
			&= P(X \leq y/a)  \\
			&= F_X(y/a)  \\
			& \mbox{differentiate} \\
		\mathbf{f_Y(y)} & \mathbf{= (1/a) \cdot f_X(y/a)}
\end{align*}

\subsection{Location and Scale}
\textbf{$Y = aX + b $}: (and $a > 0$). 
\begin{align*}
	F_Y(y) &= P(Y \leq y) \\
			&= P(aX + b \leq y)  \\
			&= P(X \leq (y-b)/a)  \\
			&= F_X((y-b)/a)  \\
			& \mbox{differentiate} \\
		\mathbf{f_Y(y)} & \mathbf{= (1/a) \cdot f_X((y-b)/a)}
\end{align*}
Visualize the uniform distribution over $[0,1]$.  When you have $W = aX+B$, you scale down so the height is now $1/a$ over $[0, a]$.  Then you shift it to $1/a$ over $[b, b+a]$.   {\tiny (10/30 lecture; last page)} \hfill \\
  \hfill \\
Example: (10/30 lecture) (??) Linear function of a normal distribution is a normal distribution.  The mean and variance drop out. (??)   \hfill \\

\subsection{Tricks for adding and multiplying random variables}

\textbf{you can get mixures of fs, Ps, and Fs and still make the connection.}
Start with converting things to Fs, then use Ps, then move back to fs if desired.  See dog eared notebook page of MT2 practice problems.  And Lecture 10/30.  {\tiny This works great for location/scale.  I'm not sure how applicable it is for R.V. addition. }\hfill \\
\hfill \\

\textbf{example: add two variables}.  X and Y are uniform on $[0,1]$.  Find pdf of $Z = X + Y$.  Note that your range for Z is [0, 2].  Expect lots of probability around $Z = 1$, and zero at $Z = 0$ and $Z = 1$ 
\begin{align*}
	F_Z(z) &= P(Z \leq z) = P(X + Y \leq z) \\
	& \mbox{\tiny use conditional probability \& independence of X \& Y} \\
	&= \int_{- \infty}^{\infty} P(X + Y \leq z \mid X = x) \cdot f_X(x)dx
\end{align*}

\section{Covariance}
{\tiny (11/4Lecture)}
Covariance is a measure of how much two random variables change together. \hfill \\
The sign of the covariance therefore shows the tendency in the linear relationship between the variables.   \hfill \\
The magnitude of the covariance is not easy to interpret.  \hfill \\
Variance is a special case of the covariance when the two variables are identical.   \hfill \\
The normalized version of the covariance, the correlation coefficient, however, shows by its magnitude the strength of the linear relation.  \hfill \\
  \hfill \\
 Recall that Var (variance; $\sigma ^2$) is $E[(X - \mu)^2]$.  And that $E[(X - \mu)] = 0$.  {\tiny (pg 156)} \hfill \\
 
 \hfill \\
$\mbox{Cov}(X,Y)=\mathbb{E}[(X-\mathbb{E}X)(Y- \mathbb{E}Y)]$  \hfill \\
If you multiply it out: $\mbox{Cov}(X,Y)=(\mathbb{E}(XY)) - (\mathbb{E}X)(\mathbb{E}Y)$  {\tiny (pg 189)}  \hfill \\
 \hfill \\
 If $X$ and $Y$ are independent, Cov$(X,Y)=0$.  Of course you can have $Cov(X,Y)=0$ without independence {\tiny (page 189)}.  \hfill \\
  \hfill \\
 Var$(X + Y) = \mathbb{E}((X+Y)^2) -  \mathbb{E}((X+Y))^2$
 $= Var(X) + 2 Cov(X,Y) + Var (Y)$. 
 If $X$, $Y$ are independent, $Var(X+Y) = Var(X) + Var (Y)$. \hfill \\

\hfill \\

\textbf{Covariance versus Variance}: 
A covariance refers to the measure of how two random variables will change together and is used to calculate the correlation between variables. 
The variance refers to the spread of the data set � how far apart the numbers are in relation to the mean, for instance. \hfill \\
\hfill \\

\textbf{$\mbox{Cov}(X,X) = \mbox{Var}(X)$}. {\tiny Midterm 2 practice problem.} \hfill \\
Because $\mbox{Cov}(X,X)=(\mathbb{E}(XX)) - (\mathbb{E}X)(\mathbb{E}X)$.  We can solve for $\mathbb{E}(XX)$ using the fact that correlation = 1 and $\rho(X,X)=\mbox{cov}(X,Y)/\sqrt{\mbox{var}(X)\mbox{var}(X)} = \mbox{cov}(X,X)/\mbox{var}(X)$.
 $1 = \rho(X,X)= \mbox{cov}(X,X)/\mbox{var}(X)$ so \textbf{$ \mbox{cov}(X,X) = \mbox{var}(X)$}.

\section{Correlation}

{\tiny (Lec. 11/4)}  Correlation: $\rho = \frac{\mbox{Cov}(X,Y)}{\sqrt{\mbox{Cov}(X)\mbox{Cov}(Y)}}$
Bounded by $-1 \leq \rho \leq 1$ \hfill \\
When is it 1 or -1?  When Var$(X-bY) = 0$.  Var $= 0$ when the probability mass is all at one point, not spread out. \hfill \\
This happens when X is a linear function of Y. \hfill \\

\section{Expectation, Variance, Covariance, and Correlation}
\textbf{Scaling \& shifting: }  \hfill \\
$E(aX + b) = aE(X) + b$  \hfill \\
$Var(aX + b) = a^2 Var(X)$  {\tiny(b doesn't matter.  Squared is b/c Var = $\sigma^2$; $\sigma$ scales w/ a so you get $a^2$)} \hfill \\
$Cov(aX + b, Y) = a \cdot Cov(X,Y)$ {\tiny(Has units, so keep factor of a.  b doesn't matter)}  
		Also: $Cov(aX + b, cY + d) = a\cdot c \cdot Cov(X,Y)$ \hfill \\
$Corr(aX + b, Y) = Corr(X, Y)$  {\tiny Use properties above to see.  Also, dimensionless so a can't be there.} \hfill \\
\hfill \\

\textbf{Cov, and Corr with itself: }  {\tiny Note that you can't have E or Var of something w/ itself: univariate. }  \hfill \\
$Cov(X, X) = Var(X)$  {\tiny From the definition of Variance. } \hfill \\
$Corr(X, X) = 1$ \hfill \\
 \hfill \\
 

\textbf{Cov, and Corr with an independent var.}  Say X, Y are independent. \hfill \\
$E(XY) = E(X)E(Y)$    {\tiny Can get there from a g(X,Y) idea.  g(X,Y) is xy.  
	Then $E(XY) = \int \int g(X,y) \cdot f_{X,Y}(x,y) =  \int \int xy \cdot f_{X,Y}(x,y) =  \int x \cdot f_X(x) \int y \cdot f_{Y}(y) $ . }   \hfill \\
$Cov(X, Y) = 0$   and  $Cov(aX, bY) = 0$  \hfill \\ \hfill \\
$Corr(X, Y) = 0$   \hfill \\
\hfill \\

\textbf{With constants:}  \hfill \\
$E(a) = aE(1) = a$  \hfill \\
$Var(a) = a^2 Var(1) = 0$  {\tiny ($=E[(a-E(a))^2] = E[(a-a)^2] = E[0] = 0$)} \hfill \\
$Cov(a, Y) = a \cdot Cov(1,Y)= a \cdot 0 = 0$ {\tiny(E((1 - E(1))) = 0.  Also Y isn't dependent on a.}  \hfill \\
$Corr(a, Y) =  0 $ {\tiny But it doesn't really make sense to talk about corr w/ constant.  Lec 11/4 }  \hfill \\
\hfill \\

%%%%%%%%%%%%%%%%%%%%%%%%%%%%%%%%%%%%%%%%%%%%%%

\section{Bernoulli}
{\tiny Wikipedia:}  The probability distribution of a random variable which takes the value 1 with success probability of p and the value 0 with failure probability of q=1-p. 
It can be used to represent a coin toss where 1 and 0 would represent "head" and "tail" (or vice versa), respectively. In particular, unfair coins would have $p \neq 0.5$.

\textbf{Things based on Bernoulli}:
\begin{itemize}
	\item The \textbf{inter-arrival time}  % http://www.math.uah.edu/stat/poisson/Introduction.html
	\item The \textbf{negative binomial distribution} (with stopping parameter n and success parameter p).
	\item The \textbf{binomial distribution} counts the number of events in a number of trials. 
\end{itemize}

\subsection{Bernoulli relation to Bionomial}
"A success/failure experiment is also called a Bernoulli experiment or Bernoulli trial; when n = 1." {\tiny (Wikipedia)}  \hfill \\
 \hfill \\
Later: "The Bernoulli distribution is a special case of the binomial distribution, where n = 1. 
Symbolically, X ~ B(1, p) has the same meaning as X ~ Bern(p). 
Conversely, any binomial distribution, B(n, p), is the distribution of the sum of n Bernoulli trials, Bern(p), each with the same probability p."  {\tiny (Wikipedia)}

Also, recall that as n gets large but p is fixed $\frac{X-np}{\sqrt{n p (1-p)}}$ is approx N(0,1).  Below, if Bernoulli n gets large and p get small you get Poisson. 

\subsection{Bernoulli relation to Geometric}
 {\tiny (11/20 Lec)}.  Independent trials, each with probability P(success) = p.  W = number of trials to first success. 
\begin{align*}
	 P(W > k) &= P(\mbox{0 successes in k trials}) \\
	 	& = (1-p)^k  \\
	\mbox{Now to get W=k} & \mbox{ we subtract b/c discrete (like differentiating)} \\
	\mbox{Note: subtracting the} & \mbox{ bigger number of events from the smaller} \\
	P(W = k) &= P(W > k - 1) - P(W > k)\\
		& = (1-p)^{k-1} - (1-p)^k \\
		& = (1-p)^{k-1}[1-(1-p)] \\
	\mathbf{P(W = k)} & = \mathbf{(1-p)^{k-1} \cdot p}  \\
	\mbox{i.e.} & \mbox{ Geom(P), mean (1/p)}
\end{align*} 
Geometric also has forgetting property: 
$P(W > k+m | W > k) = \frac{P(W > k + m)}{P(W > k)} = \frac{(1-p)^{k+m}}{(1-p)^k} = (1-p)^m = P(W>m)$ \hfill \\
\hfill \\

Geometric and exponential are the discrete/continuous analogs.  Expo w/ parameter $\lambda$ has mean $1/\lambda$.  Geom(p) has mean $1/p$.  Don't do continuity correction on exponential.   {\tiny (11/20 Lec)} 
$\mathbb{E}(W) = 1/p$ and Var(W)=$\frac{1-p}{p^2}$  \hfill \\
\hfill \\


\textbf{Number of \underline{failures} before 1st success:}   {\tiny (11/20 Lec)}\hfill \\
W* = number of failures before 1st success. 
Same pdf, just slightly different notation.  W* = W - 1,  where $P(W=k) = (1-p)^{k-1}p, k = 1, 2, \dots$ \hfill \\
Then $P(W*=k) = P(W=k+1) = (1-p)^k p$   \hfill \\
$\mathbf{ P(W^* = k) = (1-p)^k p}$, \hfill \\
$\mathbb{E}(W^*) = 1/p - 1 = \frac{1-p}{p}$ Var(W*)=$\frac{1-p}{p^2}$  = Var(W)  \hfill \\
 \hfill \\

\textbf{Number of failures before $r^{th}$ success:}   {\tiny (11/20 Lec)}\hfill \\
$W_r^*$ = number of failures before $r^{th}$ success.   $W_r^* = 0, 1, 2, \dots$  \hfill \\
$\mathbb{E}(W_r^*) = r/p - r = r(1-p)/p$ \hfill \\
Var($W_r^*) = r(1-p)/p^2$ = Var($W_r$) \hfill \\
Subtracting shifts mean but not variance. 




%%%%%%%%%%%%%%%%%%%%%%%%%%%%%%%%%%%%%%%%%%%%%%

\section{Poisson process}
\textbf{Things based on Poisson process}:
\begin{itemize}
	\item exponential 
\end{itemize}

Internet tidbids: 
\begin{itemize}
	\item Several important probability distributions arise naturally from the Poisson process�the Poisson distribution, the exponential distribution, and the gamma distribution. % http://www.math.uah.edu/stat/poisson/
	\item In some sense, the Poisson process is a continuous time version of the Bernoulli trials process  % http://www.math.uah.edu/stat/poisson/Introduction.html
\end{itemize}

\textbf{"Events happening randomly and independently over time at rate $\lambda$"}  {\tiny (11/18 Lec)} \hfill \\
In a very small time interval of length h:  \hfill \\
*P(0 events) $= 1 - \lambda h$  \hfill \\
*P(1 events) $= \lambda h$  \hfill \\
*P( $>$ 1 event) $\approx 0$  \hfill \\
Divide time from 0 to s up into K bits of length h.  K is large, h is small. $Kh \equiv s$ (they add up). \hfill \\
The number of events is binomial.  And, in each interval event or not is Bern($\lambda h)$.  \hfill \\
Bin$(K, \lambda h) \approx $ Poisson($K \lambda h$) $\cong$ Poisson($\lambda s$).  {\tiny (11/18 Lec)} \hfill \\
(Her s appears to be equal to book's T.) \hfill \\
\hfill \\
B(20,000, 1/2,000) $\approx$ Poisson (20,000/2000) = Poisson (10). {\tiny (11/20 Lec)} \hfill \\
\hfill \\

\textbf{Birthday example} \hfill \\
First note that P(0 pairs) = $(365 \cdot 364 \cdot \dots 335)/365^{31} = 0.2695$ \hfill \\
Without Poisson:  \hfill \\
1a. Derive probability of pairs.  For 31 students, there are ${31 \choose 2} = 465$  pairs.  \hfill \\
2a. Expected number of pairs: 465/365 = 1.274 \hfill \\
With Poission: \hfill \\
* Note large n, small p.  n = 465, p = 1/365.  So $\lambda$ = np = 465/365.  \hfill \\
* Note that pairs are not quite independent.  (see notes). \hfill \\
But, P(X=0) = $e^{- \lambda} = e^-1.274 = 0.279$.  Close to true answer above. \hfill \\
If you want $P(X >2),$ do $1 - P(X=0) - P(X=1) = 1 - \mu^0*e^{-\mu}/0! - \mu^1*e^{-\mu}/1! = 0.364$ 
Note that we don't have time in the framework of the problem; we aren't waiting for busses.  \hfill \\
See notes for P(3 people share).  n is large and p is smaller, so Poisson a better approximation.  \hfill \\
 \hfill \\
 
 \textbf{Adding Poisson}.  If $s_1$ and $s_2$ are non-overlapping time intervals and $X_1 = Po(\lambda s_1)$ and $X_2 = Po(\lambda s_2)$ then $X_1$ + $X_2$ = total number of events in times s = $s_1 + s_2$ then $X_1$ + $X_2 = Po(\lambda s)$.
 And if you sum up n $X_i$s as Y, by CLT $frac{Y - n \mu}{\sqrt(n \mu)}$ is approx N(0,1).  
  {\tiny (11/20 Lec)}  \hfill \\
  ?? What if your lambdas are different?  \hfill \\

%%%%%%%%%%%%%%%%%%%%%%%%%%%%%%%%%%%%%%%%%%%%%%

\section{Poisson Distribution}
One of the distributions that you can derive from the Poisson process.  -- Erick.   \hfill \\

% http://individual.utoronto.ca/zheli/poisson.pdf
The Poisson distribution is a discrete probability distribution that expresses
the probability of a number of events occurring in a fixed period of time if
these events occur with a known average rate and independently of the time
since the last event.

PMF: $\frac{\lambda^k}{k!} e^{-\lambda}$  \hfill \\

$\!f(k; \lambda)= \Pr(X{=}k)= \frac{\lambda^k e^{-\lambda}}{k!}$  \hfill \\

$\lambda=\operatorname{E}(X)=\operatorname{Var}(X)$  {\tiny (Wikipedia)} \hfill \\
Mean = variance.  $Var(X) = \mu$  {\tiny (Lec. 11/18)} \hfill \\
 \hfill \\
 
If Binomial n is large but p is small such that $np = \mu$ then $E(X) = np = \mu$ and Var$(X) = n \cdot p(1-p) = \mu (1-p) \approx \mu$  {\tiny (Lec. 11/18)}

{\tiny Wikipedia:} A \textbf{discrete} probability distribution that expresses the probability of a given number of events occurring in a fixed interval of time and/or space if these events occur with a known average rate and independently of the time since the last event.  \hfill \\
Can get the average number of events per time; that's $\lambda$.  Then you can calculate the probability of getting, say, 10 events in that time span. 

Relation to Bernoulli:  If the events are rare, each event is Bernoulli. 
Notes: \hfill \\
* distribution on nonnegative integers {\tiny (11/18 Lec)} \hfill \\
* doesn't make sense to scale/shift the Poission  {\tiny (11/18 Lec)} \hfill \\
\hfill \\

 Book's 3 formulas for Poisson (pg 232):  \hfill \\
$p_X(k) = e^{-np}\frac{(np)^k}{k!}$  \hfill \\
$p_X(k) = e^{-\lambda} \frac{\lambda^k}{k!}$  \hfill \\
$p_X(k) = e^{-\lambda T}\frac{(\lambda T)^k}{k!}$  \hfill \\
Note that $np = \lambda$ \hfill \\
We derived $P(X=k) = \frac{\mu ^k}{k!} e^{-\mu}$ in 11/18 Lecture.  From Binomial as N large but p small. \hfill \\

\subsection{Adding Poissons}
{\tiny (11/20 Lec)}  If you have two non-overlapping time intervals $S_1$ and $S_2$ ($S = S_1 + S_2$) then they are independent because they are non overlapping.  
Say $X_1 =  Po(\lambda S_1)$ and $X_2 =  Po(\lambda S_2)$.  
Then $X_1 + X_2$ events in time $S = S_1 + S_2$ is Po($\lambda S$).    \hfill \\
In general for independent $X_i \sim Po(\lambda S_i)$, $\sum_{1}^{k} X_i$ is Poisson with mean $\lambda(S_1 + S_2 + \dots S_k)$.    \hfill \\
Similarly if $X_1 + \dots + X_n$ are independent Po($\mu$), $Y = \sum_{1}{n}X_i$ is Poisson with mean $n \mu$ and variance  $n \mu$.  So b the Central Limit Theorem $\frac{Y - n \mu}{\sqrt(n \mu)}$ is approx N(0,1).    \hfill \\

\subsection{Adding Poissons and knowing the total number}
{\tiny (Bus example from 11/25)}.  
If you know the number of red busses and blue busses is n, what is the probability that the number of red busses is k? 
Each process is Po$(\lambda_i t)$ and independent.  The total is Po($\sum \lambda_i t$).  
\begin{align*}
	 P(\mbox{\# of red} = k | & \mbox{\# of red \& blue} = n)  \\
	 	&= P(R = k | R + B = n) \\
		& = \frac{P(R=k \cap B + R = n)}{P(R+B = n)} \\
		& = \frac{P(R=k \cap B = n-k)}{P(R+B = n)} \\
		& = \frac{P(R=k)P(B = n-k)}{P(R+B = n)} \\
		& = \frac{[\frac{exp(-\lambda t)(-\lambda t)^k}{k!}] [\frac{\exp(-\mu t)(-\mu t)^{n-k}}{(n-k)!}] }{\frac{\exp(-(\lambda + \mu) t)((\lambda + \mu) t)^{n}}{(n)!} } \\
		& \mbox{lots cancels out} \\
		&= \frac{n!}{k!(n-k)!} \frac{\lambda^k \mu^{n-k}}{(\lambda + \mu)^n} \\
		&= {n \choose k} \frac{\lambda}{\lambda + \mu} (\frac{\mu}{\lambda + \mu})^k  \\
		&= Bin(n, \frac{\lambda}{\lambda + \mu}) \\
\end{align*} 

Special case of next 1 bus:  \hfill \\
Given next bus is R or B, P(red) = $\lambda/(\lambda + \mu)$ = $3/(3+2) = 0.6$ \hfill \\
Now say there is Red at 3/hr, Blue at 2/hr, and green at 4/hr.  P(next bus green) = 4/(4+2+3) = 4/9.  Don't need to re-derive the binomial.  \hfill \\
\hfill \\

If you want to know the probability that 10 of the next 100 buses are Red, use the Binomial theorem.  If you want to know the probability that $<$ 10 of the next 100 buses are Red, use the Central Limit Theorem approximation of the binomial.  This was in HW9.  \textbf{REVIEW IT} 

\hfill \\
Given exactly 1 bus in the next t hours, \underline{when} does it occur?  Use CDF
\begin{align*}
	F_T(s) &= P(T \leq s | \mbox{one event in} (0,t) \mbox{, process rate } \lambda) \\
		&= \frac{P(\mbox{1 event in (0,s) and 0 in (s, t)})}{P(\mbox{1 in (0,t)})} \\
    		&= \frac{\frac{ \exp({-\lambda s})(\lambda s)^1 }{1!} \frac{\exp{(-\lambda (t-s))}(\lambda (t-s))^0}{0!}}  {\exp{(-\lambda t)}(\lambda t)^1/1!} \\
		&= \frac{ \exp({-\lambda s})(\lambda s) \exp({-\lambda (t-s))}}  {\exp({-\lambda t})(\lambda t)} \\
		&= s/t
\end{align*}
To get the density, differentiate.  $f_T(s) = 1/$ for $0 \leq s \leq t$ and 0 otherwise. 

\subsection{Conditioning on total \# with one process}
{\tiny (Bus example from 11/25)}.  Green buses come at rate 4/hour.  \hfill \\
What is P(4 green buses in next 15 minutes $|$ 4 in next hour)?  \hfill \\
= P(4 green buses in next 15 minutes $\cap$ 0 in 3/4 hour after)/P(4 in next hour).   \hfill \\
Use Po($\lambda$) with t = 1/4, t = 3/4, and t = 1. 
\begin{align*}
 	&= \frac{\frac{e^{-4(1/4)}(4 \cdot 1/4)^4/4!}{e^{-4\cdot (3/4)}}   }{e^{-4(1)}4^4/4!}  \\
	&= (1/4)^4. 
\end{align*}
\hfill \\ \hfill \\

{\tiny (Computer bug example from 11/25)}. \hfill \\
Assume total number of bugs in a piece of software is Po($\mu$).   \hfill \\
W = total \# of bugs.  X = total \# of bugs detected, Y = \# of bugs not detected yet.  
\begin{align*}
	P(X=x, Y=y) &= P(X=x, W=x+y) \\
		&= P(X=x \mid W=x+y)\cdot P(W=x+y) \\
		& \mbox{(used conditional probability)} \\
		&= \mbox{Bin(x+y, p) times Po(w=x+y)} \\
		& \mbox{fill in Binomial and Poisson:} \\
		&= {x + y \choose y} p^x(1-p)^y \cdot \frac{e^{-\mu} \mu^{x+y}}{(x+y)!} \\
		& \mbox{simplify, but expand exponential:} \mu p +\mu(1-p) = \mu  \\
		&= \frac{1}{x!y!}(\mu p)^x(\mu(1-p))^y \exp(-(\mu p + \mu (1-p))) \\
		&= \frac{(\mu p)^x e^{-\mu p}}{x!} \frac{(\mu(1-p))^y e^{-\mu (1-p)}}{y!} \\
		& \mbox{that's a Poisson(x) and a Poisson(Y)} \\
		& \mbox{It factorizes \& ranges are independent.  Independent!} \\
		&= P(X) \cdot P(Y) \\
		& \mbox{ where } X \sim Po(\mu p) \mbox{, and } Y \sim Po(\mu(1-p)) \\
\end{align*}
You aren't likely to have less bugs just because you found one.  
How this is different from the bus problems: you knew the total of R and B = n, or that there \underline{were} going to be 4 buses total but you don't know when they will come. \textit{Ask Erick his summary?}  \hfill \\ \hfill \\
Questions: can I jump straight to Binomial for these Poissons? 



%%%%%%%%%%%%%%%%%%%%%%%%%%%%%%%%%%
\section{Exponential}
Waiting time for Poisson Process.  Expo($\lambda$)

"The exponential distribution (a.k.a. negative exponential distribution) is the probability distribution that describes the time between events in a Poisson process, i.e. a process in which events occur continuously and independently at a constant average rate."  {\tiny (Wikipedia)}

 In many respects, the geometric distribution is a discrete version of the exponential distribution. Constant rate and memoryless. % http://www.math.uah.edu/stat/poisson/Exponential.html
 
  \hfill \\
  \textbf{PDF and CDF.}  \hfill \\
  PDF: $\lambda \exp(-\lambda x)$.  Note:  x = s = time.   \hfill \\
  CDF: $1 - \exp(-\lambda x)$.  Note:    x = s = time.   \hfill \\
  mean = expectation = $\lambda$, variance = $1/\lambda^2$    \hfill \\
  \hfill \\
 
 {\tiny (11/18 Lec)} \textbf{Waiting times in a Poisson process with rate $\lambda$.} 
 pdf: $f_T(s) = \lambda \exp(-\lambda s)$ on $s \geq 0$ and 0 otherwise.  \hfill \\
  \hfill \\
Comes from T being time to next event, $P(T > S) = $ P(0 events in time S) = P((Poisson mean $\lambda s$) =0) = $exp(\lambda s)$.  
\underline{Falls right out of Poisson Distribution with k = 0.}   \hfill \\
Has forgetting/memoryless property. $P(T > t+s | T > t) = \frac{P(T > t+s \cap T > t)}{P(T>t)}$.  
That's vanilla conditional probability.  
$= \frac{P(T > s)}{P(T>t)} = \frac{exp(-\lambda (t+s))}{exp(-\lambda t)} = \exp(-\lambda s) = P(T>s)$

\subsection{Adding Exponentials}
{\tiny (11/25 Lec, 26.5 of Wk8)}  The time to the first event is $T_1$.  This is $X_1 =$ Expo($\lambda$) with mean =  $\mathbb{E}(T_n) = 1/\lambda$ and variance $1/\lambda^2$.   \hfill \\
Now string a bunch together.  $T_n$ is the time to the $n^{th}$ event, $X_i$ is the time between the $i^{th}$ and $i-1^{th}$ event.  See diagram in 11/25 notes.     \hfill \\
$T_n = \sum_1^{n} X_i$ where $X_i$ are independent (? because times don't overlap).  \hfill \\
$\mathbb{E}(T_n) = n \mathbb{E}(T_i) = n (1/ \lambda)$   \hfill \\
Var$(T_n) = n \mbox{Var}(X_i) = n/\lambda^2$  \hfill \\
Want to get the PDF, but let's start with the CDF.   
\begin{align*}
	P(T_n \leq t) &= P( \mbox{at least n events in (0,t)} ) \\
		&= \sum_{j=n}^ \infty P( \mbox{J events in (0, t)}) \\
		&= \sum_{j=n}^\infty \frac{e^{- \lambda t (\lambda t)^j}}{j!} \\
		& \mbox{differentiate (in principle) gives the gamma distribution} \\
	f_{T_n} &=  \frac{d}{dt} (F_{T_n}(t)) 	 \\
	f_{T_n} &= G(n, \lambda) 	\mbox{ n is shape, $\lambda$ is rate}
\end{align*}
\textbf{Gamma Distribution:} probability of at least n events/successes in time t. (??)
{\tiny (Wikipedia:)} a two-parameter family of continuous probability distributions. The common exponential distribution and chi-squared distribution are special cases of the gamma distribution. 




 


	

 




